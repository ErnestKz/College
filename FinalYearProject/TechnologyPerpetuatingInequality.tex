% Created 2021-01-05 Tue 23:54
% Intended LaTeX compiler: pdflatex
\documentclass[11pt]{article}
\usepackage[utf8]{inputenc}
\usepackage[T1]{fontenc}
\usepackage{graphicx}
\usepackage{grffile}
\usepackage{longtable}
\usepackage{wrapfig}
\usepackage{rotating}
\usepackage[normalem]{ulem}
\usepackage{amsmath}
\usepackage{textcomp}
\usepackage{amssymb}
\usepackage{capt-of}
\usepackage{hyperref}
\date{\today}
\title{}
\hypersetup{
 pdfauthor={},
 pdftitle={},
 pdfkeywords={},
 pdfsubject={},
 pdfcreator={Emacs 27.1 (Org mode 9.3)}, 
 pdflang={English}}
\begin{document}

\section*{How Computer Science Research can Perpetuate Inequality and Bias}
\label{sec:org1f8af06}
\section*{Ernests Kuznecovs}
\label{sec:org5c75b39}
\section*{17332791}
\label{sec:org3d0fab9}
The question is "How Computer Science Research can Perpetuate Inequality and Bias".   

First it is necessary to understand what Computer science research is. In the wider scope of 
things, computer science is about processing information, and ways to automate the processing
of information. The more is research about how information processing can be automated and 
the better humans understand it, the wider range of possibilites begin to exist 
for how to use the technology in the real world.  

Once the possibilites for applying technologies exist, business and organisations will seek
to implement the relevant subset of the new technologies as part of their operations. This 
generates a requirement for business to gather people that are able to implement these technologies.

The profession that becomes in demand in order to implement these information processing technologies
are ICT specialists (Information and Communication Technology.    

Stats reviewd by europa.eu/eurstat makes it clear that ICT roles are increasing. [4]

So one effect of Computer Science research is an increased number ICT 
specialists in demand, how does this perpetuate inequality and bias?   

There exists a huge disproportion in terms of the ratio of men to women that
are employed as ICT specilaist, as of 2016, in the EU, out of all ICT speicalists
only 16\% were women, as mentioned in an article by the Euruopean Data Journalism
Network [1].

This is concerning as one of the most impactful aspects that determine what 
the new generation of people decide to go into are role models. The study carried
out by Marx, D. M. and J. S. Roman [2]. revealed that female students learning
about a competent female experimenter buffer womens self-appraised maths ability, which 
in turn led to successful performance on a challenging maths test.  

It seems that a lack of women specialising in ICT could limit the spectrum of
of options a young girl will think about when forming a vision of her future.   

This is concerning as the median slary for ICT specialists is much higher than
the median salary of other fields. This is another aspect in which equality
is being reduced.  

Another way in which this may create bias is that having a low number 
of women in ICT can create false perceptions of a womens ability to do certain
tasks, as was shown in a study where womens code changes to programmers code bases
were being accepted less once the it was evident the person was a woman. When the 
identity is anonymous, women have a higher percentage of accepted code changes [3].

The way in which this ties into Computer Science research is that if the rate of change
of women joining ICT (inspired by activism for example) stays lower than the rate of demand created 
from new technology being researched, then inequlaity and bias will be perpetuated.


[1] \url{https://www.europeandatajournalism.eu/eng/News/Data-news/The-ICT-sector-is-booming.-But-are-women-missing-out}   

[2] Marx, D. M. and J. S. Roman. “Female Role Models: Protecting Women’s Math Test Performance.” Personality and Social Psychology Bulletin 28 (2002): 1183 - 1193.   

[3] Terrell J, Kofink A, Middleton J, Rainear C, Murphy-Hill E, Parnin C. 2016. Gender bias in open source: Pull request acceptance of women versus men. PeerJ PrePrints 4:e1733v1 \url{https://doi.org/10.7287/peerj.preprints.1733v1}   

[4] \url{https://ec.europa.eu/eurostat/statistics-explained/index.php/ICT\_specialists\_-\_statistics\_on\_hard-to-fill\_vacancies\_in\_enterprises\#Employment\_and\_recruitment\_of\_ICT\_specialists}   
\end{document}
